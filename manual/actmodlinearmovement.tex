The \elem{locationvelocity} module can create linear motion of
objects. The velocity \attr{v} and starting position \attr{p0} can be
given in cartesian coordinates, e.g.,
\begin{lstlisting}[numbers=none]
<locationvelocity actor="/scene/obj" v="1 2 3" p0="0 0 1" t0="2"/>
\end{lstlisting}

\begin{snugshade}
{\footnotesize
\label{attrtab:linearmovement}
Attributes of element {\bf linearmovement}\nopagebreak

\begin{tabularx}{\textwidth}{lXl}
\hline
name & description (type, unit) & def.\\
\hline
\hline
\indattr{actor} & pattern to match actor objects (string array) & \\
\hline
\indattr{p0} & start position at time t0 (pos, m) & 0 0 0\\
\hline
\indattr{t0} & start time t0 (double, s) & 0\\
\hline
\indattr{v} & velocity vector (pos, m/s) & 1 1 0\\
\hline
\end{tabularx}
}
\end{snugshade}

All variables can be controlled via OSC; the \attr{actor} attribute is
used as path prefix. In the example above this would result in these
OSC variables:
\begin{verbatim}
/scene/obj/v/x (d)
/scene/obj/v/y (d)
/scene/obj/v/z (d)
/scene/obj/p0/x (d)
/scene/obj/p0/y (d)
/scene/obj/p0/z (d)
/scene/obj/t0 (d)
/scene/obj/vpt (ddddddd)
\end{verbatim}
Note that only setting the last OSC variable \verb!/scene/obj/vpt! ensures an atomic operation of setting the variables. If you set it variable by variable, you may get undefined (and possibly extreme) intermediate values.

\definecolor{shadecolor}{RGB}{236,236,255}\begin{snugshade}
{\footnotesize
\label{osctab:tascarmodlinearmovement}
OSC variables:
\nopagebreak

\begin{tabularx}{\textwidth}{llllX}
\hline
path & fmt. & range & readable & description\\
\hline
\attr{/.../p0/x} & f & start x-position at time t0 in m & yes & \\
\attr{/.../p0/y} & f & start y-position at time t0 in m & yes & \\
\attr{/.../p0/z} & f & start z-position at time t0 in m & yes & \\
\attr{/.../t0} & f & reference session time in s & yes & \\
\attr{/.../v/x} & f & velocity in x-direction in m/s & yes & \\
\attr{/.../v/y} & f & velocity in y-direction in m/s & yes & \\
\attr{/.../v/z} & f & velocity in z-direction in m/s & yes & \\
\attr{/.../vpt} & ddddddd &  & no & \\
\hline
\end{tabularx}
}
\end{snugshade}
\definecolor{shadecolor}{RGB}{255,230,204}

