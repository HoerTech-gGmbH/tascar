The module {\bf pos2osc} sends position and orientation of \tascar{}
objects as OSC message. This can be used to control objects in
computer graphics tools. Example:
\begin{lstlisting}[numbers=none]
<pos2osc url="osc.udp://localhost:9999/" pattern="/*/cg_*" mode="2"/>
\end{lstlisting}
The \attr{pattern} attribute specifies the object (or objects) whose geometry information will be sent.
%
In the example above all objects, whose name starts with \verb!cg_! will send geometry data.
%
The Euler-angles are sent in degrees, Cartesian coordinates in meter. 

\definecolor{shadecolor}{RGB}{255,230,204}\begin{snugshade}
{\footnotesize
\label{attrtab:pos2osc}
Attributes of element {\bf pos2osc}\nopagebreak

\begin{tabularx}{\textwidth}{lXl}
\hline
name & description (type, unit) & def.\\
\hline
\hline
\indattr{addparentname} & When sending sound vertex positions, add parent name to vertex name (bool) & false\\
\hline
\indattr{avatar} & Name of object to be controlled (for control of game engines) (string) & \\
\hline
\indattr{ignoreorientation} & Ignore delta-orientation of source, send zeros instead (bool) & false\\
\hline
\indattr{lookatlen} & Duration of look-at animation (for control of game engines) (double, s) & 1\\
\hline
\indattr{mode} & Message format mode (uint32) & 0\\
\hline
\indattr{name} & Default name used in OSC variables (string) & pos2osc\\
\hline
\indattr{orientationname} & Name for orientation variables (string) & /headGaze\\
\hline
\indattr{oscale} & Scaling factor for orientations (float) & 1\\
\hline
\indattr{pattern} & Pattern of TASCAR object names; see actor module documentation for details. (string array) & /*/*\\
\hline
\indattr{sendsounds} & Send also position of sound vertices (modes 2 and 3 only) (bool) & false\\
\hline
\indattr{skip} & Skip frames to reduce network traffic (uint32) & 0\\
\hline
\indattr{taumin} & Minimum period time between two transmissions. (float, s) & 0\\
\hline
\indattr{threaded} & Use additional thread for sending data to avoid blocking of real-time audio thread (bool) & true\\
\hline
\indattr{transport} & Send only while transport is rolling (bool) & true\\
\hline
\indattr{triggered} & Send data only when triggered via OSC (bool) & false\\
\hline
\indattr{ttl} & Time to live of OSC multicast messages (uint32) & 1\\
\hline
\indattr{url} & Target URL (string) & {\tiny osc.udp://localhost:9999/}\\
\hline
\end{tabularx}
}
\end{snugshade}


The operation modes are:

{\small
\begin{tabular}{cl}
mode & send to\dots                                                                 \\
\hline
0    & /scene/name/pos (x,y,z) and /scene/name/rot (Euler-Z,Euler-Y,Euler-X)        \\
1    & /scene/name/pos (x,y,z,Euler-Z,Euler-Y,Euler-X)                              \\
2    & /tascarpos (/scene/name,x,y,z,Euler-Z,Euler-Y,Euler-X)                       \\
3    & /tascarpos (name,x,y,z,Euler-Z,Euler-Y,Euler-X)                              \\
4    & /avatar /lookAt x,y,z,lookatlen                                              \\
5    & /avatar Euler-Z                                                              \\
6    & /avatar <orientationname> Euler-Y, Euler-Z, Euler-X (delta orientation only) \\
7    & /avatar <orientationname> Euler-Y, Euler-Z, Euler-X                          \\
8    & /avatar Euler-Y, Euler-Z, Euler-X (delta orientation only, degree)           \\
9    & /avatar <orientationname> Euler-X, Euler-Y, Euler-Z (delta orientation only) \\
11   & /avatar/<objname> x, y, z, Euler-Z, Euler-Y, Euler-X                         \\
\end{tabular}
}
