The \elem{rotator} module can create parametric rotation of objects
around the $z$-axis. Four modes are supported, {\em linear}
(\attr{mode="0"}, default), {\em sigmoid} (\attr{mode="1"}), {\em
  cosine} (\attr{mode="2"}), and {\em free} (\attr{mode="3"}).

\definecolor{shadecolor}{RGB}{255,230,204}\begin{snugshade}
{\footnotesize
\label{attrtab:rotator}
Attributes of element {\bf rotator}\nopagebreak

\begin{tabularx}{\textwidth}{lXl}
\hline
name & description (type, unit) & def.\\
\hline
\hline
\indattr{actor} & pattern to match actor objects (string array) & \\
\hline
\indattr{mode} & Operation mode (uint32, 0|1|2|3) & 0\\
\hline
\indattr{phi0} & Start angle (sigmoid/cosine movement) (double, deg) & -90\\
\hline
\indattr{phi1} & End angle (sigmoid/cosine movement) (double, deg) & 90\\
\hline
\indattr{t0} & Starting time (double, s) & 0\\
\hline
\indattr{t1} & End time (sigmoid/cosine movement) (double, s) & 1\\
\hline
\indattr{w} & Angular velocity (double, deg/s) & 10\\
\hline
\end{tabularx}
}
\end{snugshade}

\definecolor{shadecolor}{RGB}{236,236,255}\begin{snugshade}
{\footnotesize
\label{osctab:tascarmodrotator}
OSC variables:
\nopagebreak

\begin{tabularx}{\textwidth}{llllX}
\hline
path & fmt. & range & r. & description\\
\hline
\attr{/.../mode} & i &  & yes & \\
\attr{/.../phi0} & f &  & yes & \\
\attr{/.../phi1} & f &  & yes & \\
\attr{/.../t0} & f &  & yes & \\
\attr{/.../t1} & f &  & yes & \\
\attr{/.../w} & f &  & yes & \\
\hline
\end{tabularx}
}
\end{snugshade}
\definecolor{shadecolor}{RGB}{255,230,204}


Examples:

\paragraph{Linear rotation}

\begin{lstlisting}[numbers=none]
  <rotator mode="0" t0="2" w="10" actor="/*/out"/>
\end{lstlisting}

\begin{equation}
O_z = w (t-t_0)
\end{equation}

\paragraph{Sigmoid rotation}

\begin{lstlisting}[numbers=none]
  <rotator mode="1" t0="2" t1="5" phi0="-120" phi1="10" actor="/*/out"/>
\end{lstlisting}

\begin{equation}
O_z = \varphi_0+\frac{\varphi_1-\varphi_0}{1+e^{-2\pi (t-0.5(t_0+t_1))/(t_1-t_0)}}
\end{equation}

\paragraph{Cosine rotation}

\begin{lstlisting}[numbers=none]
  <rotator mode="2" t0="2" t1="5" phi0="-120" phi1="10" actor="/*/out"/>
\end{lstlisting}

\begin{equation}
O_z = \left\{\begin{array}{ll}
\varphi_0 & t<t_0\\
\varphi_0+\frac12(\varphi_1-\varphi_0)(1-\cos(\pi \frac{t-t_0}{t_1-t_0})) & t_0\le t \le t_1\\
\varphi_1 & t_1 < t
\end{array}\right.
\end{equation}

\paragraph{Free rotation}

Same as linear, but the rotation phase is continuously incremented
independent of the transport time.
