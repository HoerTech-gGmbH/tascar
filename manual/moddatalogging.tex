The data logging module allows logging of OSC messages and LSL streams
together with the \tascar{} timeline.
%
Both OSC and LSL are protocols commonly used to send information
between devices, such as motion capture devices or bioamplifiers.
%
For example, you can collect data from such sensors that are used
during an experiment.
%
If the sensor sends data to the computer either in OSC or LSL
protocol, you can record this data using the \elem{datalogging}
module:

Example:
\begin{lstlisting}[numbers=none]
<datalogging port="9998" multicast="" fileformat="matcell" outputdir="${HOME}">
  <osc path="/sensor1/pos" size="3"/>
  <osc path="/sensor1/rot" size="3"/>
  <osc path="/sensor3" size="4"/>
  <oscs path="/msg"/>
  <lsl predicate="name='EEGamp'"/>
</datalogging>
\end{lstlisting}

To record the data sent by a device as a series of OSC messages, we
have to specify the path to the OSC variable associated with the
sensor:
%
\begin{lstlisting}[numbers=none]
  <osc path="/sensor1/pos" size="3"/>
\end{lstlisting}
%
The \attr{size} attribute refers to the dimension of the OSC variable.
%
The attribute \attr{ignorefirst} can be used to hide the first channel
in the display, which can be useful if the first channel contains time
values or other control data.

To record the data sent by a device as an LSL stream, we need to
specify the name of the LSL stream associated with the sensor:
\begin{lstlisting}[numbers=none]
  <lsl predicate="name='EEGamp'" tctimeout="2"/>
\end{lstlisting}
The attribute \attr{tctimeout} is the maximal time used to measure the
time correction values between sender and receiver. The
attribute \attr{required} can be set to ``false'', to allow for
starting \tascar{} without requiring all LSL streams to be
available. The stream will not be recovered later during the session.

Text data (e.g., trigger messages) can be recorded from LSL, or from osc with the \elem{oscs} element:
\begin{lstlisting}[numbers=none]
  <oscs path="/msg"/>
\end{lstlisting}


\begin{snugshade}
{\footnotesize
\label{attrtab:datalogging}
Attributes of element {\bf datalogging}\nopagebreak

\begin{tabularx}{\textwidth}{lXl}
\hline
name & description (type, unit) & def.\\
\hline
\hline
\indattr{controltransport} & Control transport with recording session control (bool) & true\\
\hline
\indattr{displaydc} & Display DC components (bool) & true\\
\hline
\indattr{fileformat} & File format, can be either ``mat'', ``matcell'' or ``txt'' (string) & matcell\\
\hline
\indattr{lsltimeout} & Number of seconds to scan for LSL streams (double, s) & 10\\
\hline
\indattr{multicast} & OSC multicasting address (string) & \\
\hline
\indattr{outputdir} & Data output directory (string) & \\
\hline
\indattr{port} & OSC port, or empty to use session server (string) & \\
\hline
\indattr{srv\_proto} & Server protocol, UDP or TCP (string) & UDP\\
\hline
\indattr{usetransport} & Record only while transport is rolling (bool) & false\\
\hline
\end{tabularx}
}
\end{snugshade}

The window size and position of the datalogging GUI can be controlled
with the attributes \attr{x}, \attr{y}, \attr{w} and \attr{h}.

% some more documentation of attributes...

Depending on the content of the \attr{fileformat} variable, the
storage format differs: In the \verb!mat! file format, each variable
is stored as a matrix under the variable name. This means that it is
not possible to record two streams with the same variable name. To
work around this problem, the \verb!matcell! file format can be
used. Here the data is stored in a cell array, with one entry for each
variable. Each entry contains a structure, with a \attr{name} field,
a \attr{data} field and for LSL variables some additional stream
information.

\subsubsection*{Data logging, session time and lab streaming layer}

The data logging can record two types of streams: OSC based floating
point values (\elem{osc}), and LSL based floating point streams
(\elem{lsl}).
%
For OSC messages, the first row of the data matrix contains the
session time $t_\text{session}$ at which the data packet arrived.
%
The underlying function from the jack audio connection kit,
\verb!jack_get_current_transport_frame!, is used to get a high
resolution estimate of the current session time.
%
For LSL streams, the situation is more complex, since LSL provides an
own method of time stamping.
%
Here, the second row in the data matrix contains the original LSL time
stamps of the remote sender, $t_\text{lsl,remote}$.
%
Since the data is processed in chunks, it is not possible to use the
arrival time as a session time stamp.
%
Instead, the clock difference between the local LSL clock and the
remote LSL clock $\Delta_\text{stream}$ is measured at the beginning
and also at the end of each recording session, using the LSL function
\verb!lsl_time_correction!, i.e., the local LSL clock minus the remote
clock, $\Delta_\text{stream}=t_\text{lsl,local}-t_\text{lsl,remote}$.
%
Additionally, upon each update of the local session time, i.e., upon
each processing cycle, the difference between the session time and the
local LSL time,
$\Delta_\text{session}=t_\text{session}-t_\text{lsl,local}$ is
measured.
%
The combination of $\Delta_\text{session}$ and $\Delta_\text{stream}$
is used to convert remote LSL time stamps into session time stamps: the estimated session time at time of sending the sample, $\tilde{t}_\text{session}$ is
\begin{equation}
\tilde{t}_\text{session} = t_\text{lsl,remote} + \Delta_\text{stream} + \Delta_\text{session}
\end{equation}
$\Delta_\text{stream}$ is the value which was measured at the
beginning of a recording session. $\tilde{t}_\text{session}$ is the
time stamp which is stored in the first row of the LSL data matrix.

Clock drift may occur between clocks. The drift between the local LSL
clock $t_\text{lsl,local}$ and the audio clock (basis of
$t_\text{session}$) is continuously compensated by the measures of
$\Delta_\text{session}$. The drift between the local LSL time
$t_\text{lsl,local}$ and the remote LSL time $t_\text{lsl,remote}$ can
be compensated offline by taking the difference between
$\Delta_\text{stream}$ at the beginning and the end of a recording
session, which are both stored in the datalogging file for each LSL
stream. Thus the drift-compensated estimated session time
$\hat{t}_\text{session}$ is
\begin{equation}
\hat{t}_\text{session} = \tilde{t}_\text{session} + \frac{t_\text{lsl,local}-t_\text{lsl,local,start}}{t_\text{lsl,local,end}-t_\text{lsl,local,start}} (\Delta_\text{stream,end}-\Delta_\text{stream,start}).
\end{equation}

Some sensors (e.g., the ESP-based IMU/EOG sensor of the Gesture lab in
University of Oldenburg), synchronize the sensor clock with the
(remote) LSL clock only upon initialization. This causes the problem,
that the clock drift reported by $\Delta_\text{stream}$ is not related
to the clock drift between the sensor and the session time. To
overcome this problem, the \elem{espheadtracker} glabsensor submodule
(see section \ref{sec:espheadtracker}) sends a local difference
$\Delta_\text{sensor}=t_\text{lsl,remote}-t_\text{sensor}$ as an LSL
stream. This data contains drift as well as jitter caused by the WiFi
transmission. The sensor drift can be estimated by a linear fit to
this data. The linear fit of $\Delta_\text{sensor}$ needs to be added
to $\hat{t}_\text{session}$ of the data of the LSL streams
corresponding to this sensor.

