\index{light control}\index{DMX}\index{artnetDMX}\index{Entec openDMX}

The module has these attributes:
\begin{snugshade}
{\footnotesize
\label{attrtab:lightctl}
\begin{tabularx}{\textwidth}{lXl}
\hline
name                     & description (type, unit)                                                & def. \\
\hline
\hline
\indattr{fps}            & Frames per second (double, Hz)                                          & 30   \\
\hline
\indattr{universe}       & DMX universe (uint32)                                                   & 0    \\
\hline
\indattr{driver}         & Driver name (string, ``artnetdmx'', ``opendmxusb'', or ``osc'')         &      \\
\hline
\indattr{hue\_warp\_rot} & Hue warping rotation (double, deg)                                      & 0    \\
\hline
\indattr{hue\_warp\_x}   & Hue warping x offset (double)                                           & 0    \\
\hline
\indattr{hue\_warp\_y}   & Hue warping y offset (double)                                           & 0    \\
\hline
\indattr{rawsrvchannels} & Number of channels to receive as RAW DMX (uint32)                       & 0    \\
\hline
\indattr{rawsrvhost}     & multicast address for raw DX OSC server (string)                        &      \\
\hline
\indattr{rawsrvpath}     & Path for raw DMX OSC server, empty for no raw DMX OSC server (string)   &      \\
\hline
\indattr{rawsrvport}     & Port of raw DMX OSC server, or empty to use session OSC server (string) &      \\
\hline
\indattr{rawsrvproto}    & Protocol of raw DMX OSC server (string)                                 & UDP  \\
\hline
\end{tabularx}
}
\end{snugshade}

Additional attributes of ``artnetdmx'' driver:
\begin{snugshade}
{\footnotesize
\label{attrtab:lightctl}
\begin{tabularx}{\textwidth}{lXl}
\hline
name               & description (type, unit)                   & def.      \\
\hline
\hline
\indattr{hostname} & Hostname of ArtnetDMX receiver (string)    & localhost \\
\hline
\indattr{port}     & Port number of ArtnetDMX receiver (uint32) & 6454      \\
\hline
\end{tabularx}
}
\end{snugshade}

\begin{snugshade}
{\footnotesize
\label{attrtab:lightctl}
Additional attributes of ``opendmxusb'' driver

\begin{tabularx}{\textwidth}{lXl}
\hline
name             & description (type, unit) & def.         \\
\hline
\hline
\indattr{device} & Device name              & /dev/ttyUSB0 \\
\hline
\end{tabularx}
}
\end{snugshade}

\begin{snugshade}
{\footnotesize
\label{attrtab:lightctl}
Additional attributes of ``osc'' driver

\begin{tabularx}{\textwidth}{lXl}
\hline
name                  & description (type, unit)                        & def.      \\
\hline
\hline
\indattr{hostname}    & Hostname of OSC destination (string)            & localhost \\
\hline
\indattr{port}        & Port number of OSC destination (uint32)         & 9000      \\
\hline
\indattr{path}        & Destination path (string)                       & /dmx      \\
\hline
\indattr{maxchannels} & Maximum number of channels to transmit (uint32) & 512       \\
\hline
\end{tabularx}
}
\end{snugshade}


One or more \elem{lightscene} elements can be defined, with these
attributes:
\definecolor{shadecolor}{RGB}{255,230,204}\begin{snugshade}
{\footnotesize
\label{attrtab:lightscene}
Attributes of element {\bf lightscene}\nopagebreak

\begin{tabularx}{\textwidth}{lXl}
\hline
name & description (type, unit) & def.\\
\hline
\hline
\indattr{channels} & Number of DMX channels per fixture (uint32) & 3\\
\hline
\indattr{layout} & name of speaker layout file (string) & \\
\hline
\indattr{master} & undocumented (float) & 1\\
\hline
\indattr{method} & undocumented (string) & \\
\hline
\indattr{mixmax} & undocumented (bool) & false\\
\hline
\indattr{name} & Scene name (string) & lightscene\\
\hline
\indattr{objects} & Pattern of objects to track (string array) & \\
\hline
\indattr{objval} & DMX value of objects (float array) & \\
\hline
\indattr{objw} & weight of objects (float array) & \\
\hline
\indattr{parent} & Name of parent object for relative position measurement (string) & \\
\hline
\indattr{sendsquared} & Send squared values for smoother intensity fades (bool) & false\\
\hline
\indattr{usecalib} & Use calibrated values instead of raw values (bool) & true\\
\hline
\end{tabularx}
}
\end{snugshade}
%%
%%\begin{tscattributes}
%%\indattr{name}        & lightscene name                                                               \\
%%\indattr{objects}     & Tracking source object name pattern                                           \\
%%\indattr{parent}      & light center parent object name                                               \\
%%\indattr{channels}    & number of channels per fixture                                                \\
%%\indattr{master}      & Master control                                                                \\
%%\indattr{usecalib}    & Use calibrated (true, default) or raw (false) DMX values                      \\
%%\indattr{sendsquared} & Send squared values (default: false)                                          \\
%%\indattr{method}      & Rendering method (nearest, raisedcosine, sawleft, sawright, rect)             \\
%%\indattr{mixmax}      & mix light based on maximum between fixture and object values (default: false) \\
%%\indattr{objval}      & Starting object color values.                                                 \\
%%\indattr{objw}        & Starting object width values.                                                 \\
%%\indattr{layout}      & fixture layout file name (optional)                                           \\
%%\end{tscattributes}

Fixtures are defined using the \indattr{fixture} element within the
\indattr{fixtures} element.
%
Syntax is the same as for speaker layout definitions, with these
additional attributes for each element:
\definecolor{shadecolor}{RGB}{255,230,204}\begin{snugshade}
{\footnotesize
\label{attrtab:fixture}
Attributes of element {\bf fixture}\nopagebreak

\begin{tabularx}{\textwidth}{lXl}
\hline
name & description (type, unit) & def.\\
\hline
\hline
\indattr{addr} & start address (uint32) & 1\\
\hline
\indattr{az} & Azimuth (double, deg) & 0\\
\hline
\indattr{el} & Elevation (double, deg) & 0\\
\hline
\indattr{label} & fixture label (string) & \\
\hline
\indattr{dmxval} & start DMX value (int32 array) & \\
\hline
\end{tabularx}
}
\end{snugshade}
For each fixture, sub-elements in the form \attr{<calib channel="0"
  in="255" out="127"/>} can be provided, to calibrate the input-output
function of the lamps. The attributes \attr{in} needs to be larger
than zero, the attributes \attr{channel} and \attr{out} need to be
larger or equal to zero.
%
Instead of linear DMX values these can be squared ($\textrm{DMX}_o=255\,\textrm{ceil}(\textrm{DMX}_i/255)^2$), to achieve constant intensity light.

Calibration tools for MATLAB/GNU Octave are available
in \verb!tascar_fixtures_calib.m!. See source code repository for more
examples.
