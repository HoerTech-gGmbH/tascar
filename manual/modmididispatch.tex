This plugins can dispatch OSC messages upon MIDI events (CC or note events). Event handlers can be registered via OSC or in the XML configuration, using the \elem{ccmsg} or \elem{notemsg} elements (see below). Parameters to the message can be added using the \elem{f v="1.234"},  \elem{i v="1"} or  \elem{s v="string"} sub-elements. Multiple event handlers for the same event can be registered. In that case all event handlers will be called. Event handler can be removed via OSC. The communication is bi-directional; MIDI events can be emitted by sending an OSC message to \verb!/mididispatch/send/cc! or  \verb!/mididispatch/send/note!.

\begin{snugshade}
{\footnotesize
\label{attrtab:mididispatch}
Attributes of element {\bf mididispatch}\nopagebreak

\begin{tabularx}{\textwidth}{lXl}
\hline
name & description (type, unit) & def.\\
\hline
\hline
\indattr{connect} & ALSA device name to connect to (string) & \\
\hline
\indattr{copyccpath} & OSC path for copied CC events (string) & /cc\\
\hline
\indattr{copynotepath} & OSC path for copied note events (string) & /note\\
\hline
\indattr{copyurl} & OSC URL to copy outgoing MIDI messages to. (string) & \\
\hline
\indattr{dumpmsg} & Dump all unrecognized messages to console (bool) & true\\
\hline
\indattr{name} & ALSA MIDI name (string) & mididispatch\\
\hline
\indattr{oscinput} & Create additional OSC inputs (bool) & false\\
\hline
\end{tabularx}
}
\end{snugshade}

\begin{snugshade}
{\footnotesize
\label{attrtab:mididispatchccmsg}
Attributes of element {\bf ccmsg}\nopagebreak

\begin{tabularx}{\textwidth}{lXl}
\hline
name & description (type, unit) & def.\\
\hline
\hline
\indattr{channel} & MIDI channel (uint32) & 0\\
\hline
\indattr{param} & MIDI CC parameter (uint32) & 0\\
\hline
\indattr{mode} & message mode, float|trigger (string) & trigger\\
\hline
\indattr{path} & OSC path (string) & \\
\hline
\indattr{min} & lower bound (float) & 0\\
\hline
\indattr{max} & upper bound (float) & 127\\
\hline
\end{tabularx}
}
\end{snugshade}


\begin{snugshade}
{\footnotesize
\label{attrtab:mididispatchccmsg}
Attributes of element {\bf notemsg}\nopagebreak

\begin{tabularx}{\textwidth}{lXl}
\hline
name & description (type, unit) & def.\\
\hline
\hline
\indattr{channel} & MIDI channel (uint32) & 0\\
\hline
\indattr{note} & MIDI note (uint32) & 0\\
\hline
\indattr{mode} & message mode, float|trigger (string) & trigger\\
\hline
\indattr{path} & OSC path (string) & \\
\hline
\indattr{min} & lower bound (float) & 0\\
\hline
\indattr{max} & upper bound (float) & 127\\
\hline
\end{tabularx}
}
\end{snugshade}

An example configuration can look like this:

\begin{lstlisting}
<modules>
  <mididispatch connect="US-2x2:US-2x2 MIDI 1" dumpmsg="true">
    <notemsg channel="0" note="0" path="/runscript"><s v="note0"/></notemsg>
  </mididispatch>
</modules>
\end{lstlisting}

\definecolor{shadecolor}{RGB}{236,236,255}\begin{snugshade}
{\footnotesize
\label{osctab:tascarmodmididispatch}
OSC variables:
\nopagebreak

\begin{tabularx}{\textwidth}{llllX}
\hline
path & fmt. & range & r. & description\\
\hline
\attr{/.../add/cc/float} & iisff &  & no & \\
\attr{/.../add/cc/float} & iisffs &  & no & \\
\attr{/.../add/cc/trigger} & iisii &  & no & \\
\attr{/.../add/cc/trigger} & iisiis &  & no & \\
\attr{/.../add/note/float} & iisff &  & no & \\
\attr{/.../add/note/float} & iisffs &  & no & \\
\attr{/.../add/note/trigger} & iisii &  & no & \\
\attr{/.../add/note/trigger} & iisiis &  & no & \\
\attr{/.../clear/launchpadaction} &  &  & no & \\
\attr{/.../del/cc/all} &  &  & no & \\
\attr{/.../del/cc} & ii &  & no & \\
\attr{/.../del/launchpadaction} & i &  & no & \\
\attr{/.../del/note/all} &  &  & no & \\
\attr{/.../del/note} & ii &  & no & \\
\attr{/.../select/launchpadaction} & s &  & no & \\
\attr{/.../send/cc} & iii &  & no & \\
\attr{/.../send/note} & iii &  & no & \\
\hline
\end{tabularx}
}
\end{snugshade}
\definecolor{shadecolor}{RGB}{255,230,204}

