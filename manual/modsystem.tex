Start system processes, e.g., to load helper programs, external decoders or video render tools.

\definecolor{shadecolor}{RGB}{255,230,204}\begin{snugshade}
{\footnotesize
\label{attrtab:system}
Attributes of element {\bf system}\nopagebreak

\begin{tabularx}{\textwidth}{lXl}
\hline
name & description (type, unit) & def.\\
\hline
\hline
\indattr{allowoscmod} & allow modifications of timed commands via OSC (bool) & false\\
\hline
\indattr{command} & command to be executed (string) & \\
\hline
\indattr{id} & undocumented (string) & system\\
\hline
\indattr{noshell} & do not use shell to spawn subprocess (bool) & true\\
\hline
\indattr{onunload} & command to be executed when unloading session (string) & \\
\hline
\indattr{relaunch} & relaunch process if ended before session unload (bool) & false\\
\hline
\indattr{relaunchwait} & Time to wait before relaunching subprocess (double, s) & 0\\
\hline
\indattr{sleep} & wait after starting the command before continuing to load session (double, s) & 0\\
\hline
\indattr{timedcmdpipe} & start timed commands using a pipe (true) or fork (false) (bool) & true\\
\hline
\indattr{timedprefix} & Prefix for timed commands added via OSC (string) & \\
\hline
\indattr{triggered} & command to be executed upon trigger signal (string) & \\
\hline
\end{tabularx}
}
\end{snugshade}

If using a shell, on Unix systems the commands are started into the background using this shell command line:
\begin{verbatim}
sh -c "cd sessionpath;command >/dev/null & echo \$!"
\end{verbatim}

