\definecolor{shadecolor}{RGB}{255,230,204}\begin{snugshade}
{\footnotesize
\label{attrtab:receiverhrtf}
Attributes of receiver element {\bf hrtf}, inheriting from \hyperref[attrtab:receiver]{{\bf receiver}}\nopagebreak

\begin{tabularx}{\textwidth}{lXl}
\hline
name & description (type, unit) & def.\\
\hline
\hline
\indattr{Q\_notch} & quality factor of the notch filter (float) & 2.3\\
\hline
\indattr{alphamin} & parameter which determines the depth of the high-shelf realizing the SHM (float) & 0.14\\
\hline
\indattr{alphamin\_front} & parameter which determines the depth of the second high-shelf (float) & 0.39\\
\hline
\indattr{alphamin\_up} & parameter which determines the depth of the second high-shelf (float) & 0.1\\
\hline
\indattr{angle} & Position of the ears on the sphere (float, deg) & 90\\
\hline
\indattr{c} & Speed of sound (float, m/s) & 340\\
\hline
\indattr{decorr} & Flag to use decorrelation of diffuse sounds (bool) & false\\
\hline
\indattr{decorr\_length} & Decorrelation length (float, s) & 0.05\\
\hline
\indattr{diffuse\_hrtf} & apply hrtf model also to diffuse rendering (bool) & false\\
\hline
\indattr{freq\_end} & notch center frequency at [0 0 1] (float, Hz) & 650\\
\hline
\indattr{freq\_start} & notch center frequency at \attr{startangle\_notch} (float, Hz) & 1300\\
\hline
\indattr{gaincorr} & channel-wise gain correction (float array, dB) & 0 0\\
\hline
\indattr{nf\_angles} & angles for near field filter coefficients (float array, rad) & \\
\hline
\indattr{nf\_filter} & apply near field filter to model close sources (bool) & false\\
\hline
\indattr{nf\_range} & distance for rendering near field effect (float array, m)& 0.15 3.0\\
\hline
\indattr{nf\_p11} & Numerator coefficient p11 for DC-gain of the near-field filter for each angle in \indattr{nf\_angles} (float array) & \\
\hline
\indattr{nf\_p12} & Numerator coefficient p12 for asymptotic hf-gain of the near-field filter for each angle in \indattr{nf\_angles} (float array) & \\
\hline
\indattr{nf\_p13} & Numerator coefficient p13 for the cutoff-frequency of the near-field filter for each angle in \indattr{nf\_angles} (float array) & \\
\hline
\indattr{nf\_p21} & Numerator coefficient p21 for DC-gain of the near-field filter for each angle in \indattr{nf\_angles} (float array) & \\
\hline
\indattr{nf\_p22} & Numerator coefficient p22 for asymptotic hf-gain of the near-field filter for each angle in \indattr{nf\_angles} (float array) & \\
\hline
\indattr{nf\_p23} & Numerator coefficient p23 for the cutoff-frequency of the near-field filter for each angle in \indattr{nf\_angles} (float array) & \\
\hline
\indattr{nf\_p33} & Numerator coefficient p33 for the cutoff-frequency of the near-field filter for each angle in \indattr{nf\_angles} (float array) & \\
\hline
\indattr{nf\_q11} & Denominator coefficient q11 for DC-gain of the near-field filter for each angle in \indattr{nf\_angles} (float array) & \\
\hline
\indattr{nf\_q12} & Denominator coefficient q12 for asymptotic hf-gain of the near-field filter for each angle in \indattr{nf\_angles} (float array) & \\
\hline
\indattr{nf\_q13} & Denominator coefficient q13 for the cutoff-frequency of the near-field filter for each angle in \indattr{nf\_angles} (float array) & \\
\hline
\indattr{nf\_q21} & Denominator coefficient q21 for DC-gain of the near-field filter for each angle in \indattr{nf\_angles} (float array) & \\
\hline
\indattr{nf\_q22} & Denominator coefficient q22 for asymptotic hf-gain of the near-field filter for each angle in \indattr{nf\_angles} (float array) & \\
\hline
\indattr{nf\_q23} & Denominator coefficient q23 for the cutoff-frequency of the near-field filter for each angle in \indattr{nf\_angles} (float array) & \\
\hline
\indattr{maxgain} & gain applied at [0 0 1] -- gain is 0 dB at \attr{startangle\_notch} and increases linearly (float, dB) & -5.4\\
\hline
\indattr{omega} & cut-off frequency of the high-self realizing the SHM (float, Hz) & 3100\\
\hline
\indattr{omega\_front} & cut-off frequency of the second high-self (float, Hz) & 11200\\
\hline
\indattr{omega\_up} & cut-off frequency of the second high-shelf in Hz (float, Hz) & 2125\\
\hline
\indattr{prewarpingmode} & Azimuth pre-warping mode, 0 = original, 1 = none, 2 = corrected (uint32) & 0\\
\hline
\indattr{radius} & Radius of sphere modeling the head (float, m) & 0.08\\
\hline
\indattr{sincorder} & Sinc interpolation order of ITD delay line (uint32) & 0\\
\hline
\indattr{sincsampling} & Sinc table sampling of ITD delay line, or 0 for no table. (uint32) & 64\\
\hline
\indattr{startangle\_front} & the second high-shelf, e.g. to model pinna shadow effect, is applied when the angle with respect to front direction [1 0 0] is larger than \attr{startangle\_front} (float, deg) & 0\\
\hline
\indattr{startangle\_notch} & notch filter to model concha notch is applied if angle with respect to up direction [0 0 1] is smaller than \attr{startangle\_notch} (float, deg) & 102\\
\hline
\indattr{startangle\_up} & the third high-shelf which models the shadow effect of the torso is applied when the angle with respect to up direction [0 0 1] is larger than \attr{startangle\_up} (float, deg) & 135\\
\hline
\indattr{thetamin} & angle with respect to the position of the ears at which the maximum depth of the high-shelf realizing the SHM is reached (float, deg) & 160\\
\hline
\end{tabularx}
}
\end{snugshade}
