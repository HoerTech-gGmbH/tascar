\definecolor{shadecolor}{RGB}{255,230,204}\begin{snugshade}
{\footnotesize
\label{attrtab:sndfile}
Attributes of element {\bf sndfile}\nopagebreak

\begin{tabularx}{\textwidth}{l>{\raggedright}XX}
\hline
name & description (type, unit) & def.\\
\hline
\hline
\indattr{attribution} & attribution of license, if applicable (string) & \\
\hline
\indattr{channel} & First sound file channel to be used, zero-base (uint32) & 0\\
\hline
\indattr{channelorder} & Channel order in case of First Order Ambisonics files, ``FuMa'', ``ACN'' or ``none'' (string, FuMa|ACN|none) & \\
\hline
\indattr{length} & length of sound sample, or 0 to use whole file length (double, s) & 0\\
\hline
\indattr{level} & level, meaning depends on \attr{levelmode} (double, dB) & -inf\\
\hline
\indattr{levelmode} & level mode, ``rms'', ``peak'' or ``calib'' (string) & rms\\
\hline
\indattr{license} & license type (string) & \\
\hline
\indattr{loop} & loop count or 0 for infinite looping (uint32) & 1\\
\hline
\indattr{loopcrossexp} & exponent of von-Hann crossfade for seamless loop (float) & 1\\
\hline
\indattr{loopcrosslen} & duration of crossfade for seamless loop (float, s) & 0\\
\hline
\indattr{mute} & Load muted (bool) & false\\
\hline
\indattr{name} & Sound file name (string) & \\
\hline
\indattr{normalization} & Normalization in case of First Order Ambisonics files. (string, FuMa|SN3D) & FuMa\\
\hline
\indattr{osctriggerpath} & Target path where OSC message of final time stamp of trigger events should be sent to. (string) & /sndfile/trigger\\
\hline
\indattr{osctriggerurl} & Target URL where OSC message of final time stamp of trigger events should be sent to. (string) & \\
\hline
\indattr{position} & Start position within the scene (double, s) & 0\\
\hline
\indattr{rampend} & von-Hann ramp duration at end of sound (float, s) & 0\\
\hline
\indattr{rampstart} & von-Hann ramp duration at start of sound (float, s) & 0\\
\hline
\indattr{resample} & Allow resampling to current session sample rate (bool) & false\\
\hline
\indattr{start} & Start position within the file (double, s) & 0\\
\hline
\indattr{transport} & Use session time base (bool) & true\\
\hline
\indattr{triggered} & Use OSC variable `/loop' to trigger playback (ignores attributes `position' and `loop') (bool) & false\\
\hline
\indattr{weighting} & level weighting for RMS mode (f-weight) & Z\\
\hline
\end{tabularx}
}
\end{snugshade}
