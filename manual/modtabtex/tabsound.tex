\definecolor{shadecolor}{RGB}{255,230,204}\begin{snugshade}
{\footnotesize
\label{attrtab:sound}
Attributes of element {\bf sound} ({\hyperref[attrtab:soundcardioidmod]{cardioidmod}} {\hyperref[attrtab:sounddoor]{door}} {\hyperref[attrtab:soundfarsrc]{farsrc}} {\hyperref[attrtab:soundgeneric1storder]{generic1storder}} {\hyperref[attrtab:soundomni]{omni}}), inheriting from \hyperref[attrtab:ports]{{\bf ports}}\nopagebreak

\begin{tabularx}{\textwidth}{l>{\raggedright}XX}
\hline
name & description (type, unit) & def.\\
\hline
\hline
\indattr{airabsorption} & apply air absorption filter(bool) & true\\
\hline
\indattr{d} & distance to next sound along trajectory, or 0 for normal mode(double, m) & 0\\
\hline
\indattr{delayline} & use delayline(bool) & true\\
\hline
\indattr{gainmodel} & gain rule, valid gain models: "1/r", "1"(string) & 1/r\\
\hline
\indattr{id} & id of sound vertex(string) & 6\\
\hline
\indattr{ismmax} & maximal ISM order to render(uint32) & 2147483647\\
\hline
\indattr{ismmin} & minimal ISM order to render(uint32) & 0\\
\hline
\indattr{maxdist} & maximum distance to be used in delay lines(float, m) & 3700\\
\hline
\indattr{minlevel} & Level threshold for rendering(float, dB SPL) & -inf\\
\hline
\indattr{mute} & mute state(bool) & false\\
\hline
\indattr{name} & name of sound vertex(string) & \\
\hline
\indattr{nearfieldlimit} & distance arond 1/r source where the gain is constant(float, m) & 0.1\\
\hline
\indattr{rx} & Euler orientation (X) relative to parent(double, deg) & 0\\
\hline
\indattr{ry} & Euler orientation (Y) relative to parent(double, deg) & 0\\
\hline
\indattr{rz} & Euler orientation (Z) relative to parent(double, deg) & 0\\
\hline
\indattr{sincorder} & order of sinc interpolation in delayline(uint32) & 0\\
\hline
\indattr{size} & physical size of sound source (effect depends on rendering method)(float, m) & 0\\
\hline
\indattr{type} & source directivity type, e.g., omni, cardioid(string) & omni\\
\hline
\indattr{x} & position relative to parent(double, m) & 0\\
\hline
\indattr{y} & position relative to parent(double, m) & 0\\
\hline
\indattr{z} & position relative to parent(double, m) & 0\\
\hline
\end{tabularx}
}
\end{snugshade}
