HRTF simulation.

This receiver describes the main features of measured head related transfer
functions by using a few low-order digital filters. The parametrization is
based on the Spherical Head Model (SHM) by \citet{BrownDuda}
and includes three further low-order filters.

The SHM introduces an approach to model the head as a rigid sphere.
It includes a model for the head shadow effect as well as a method to
compute the interaural time difference.
The head shadow effect is approximated by a first-order high-shelf filter
which depth varies depending on the incident angle. The high-shelf can be
described by means of three parameters: The cut-off frequency \indattr{omega},
the angle \indattr{thetamin} at which the maximal depth of the high-shelf is
reached and the parameter \indattr{alphamin} which influences the maximal
reached depth of the high-shelf.

The Duda SHM was extended by O.\ Buttler and S.D.\ Ewert in the context
of room acoustics simulator RAZR \citep{Wendt2014b,Ewert2018}
in \citet{Buttler2018} to improve left-right, front-back, and
elevation perception:

i) a pre-warping of the azimuth angles is introduced to better match experimentally
observed interaural level differences as a function of azimuth, particularly in
the frontal region.

ii) Two further first-order high-shelf filters similarly to that which
realizes the SHM are used to model pinna
- respectively torso - shadow. These filters are as well described by
three parameters. The two parameters \indattr{alphamin\_front} and
\indattr{omega\_front} -- respectively \indattr{alphamin\_up} and
\indattr{omega\_up} -- are used in the same way as described for the SHM.
However, the third parameter \indattr{startangle\_front} -- respectively
\indattr{startangle\_up} --, which is defined with respect to a certain
reference direction (front [1 0 0] -- respectively up [0 0 1]), is used
in order to define a region of incident directions in which these filters
are applied. The maximal depth is reached at 180 degrees with respect to
the reference direction.

iii) Furthermore, a notch filter is used in order to reproduce the concha notch
which provides an important feature in order to distinguish between elevation
angles. This filter is applied in the upper hemisphere for angles smaller
than \indattr{startangle\_notch}. In order to have a smooth transition, the
gain of the notch increases linearly from 0 dB at \indattr{startangle\_notch}
to the \indattr{maxgain} for an incidence direction directly above the head.
Moreover, the center frequency is chosen to vary linearly over the range
as well. At \indattr{startangle\_notch} the center frequency is equal to
\indattr{freq\_start} as changes linearly to \indattr{freq\_end} for incidence
direction right above the head. Furthermore, the notch is described by the
quality factor \indattr{Q\_notch}.

In order to optimize the values for the filter parameters of the
original RAZR SHM-Model, the frequency response of the receiver has
been fitted to measured HRTFs of the KEMAR dummy
head \citep{Schwark2020} provided by the OlHeaD-HRTF
database~\citep{Denk2020}.

\definecolor{shadecolor}{RGB}{255,230,204}\begin{snugshade}
{\footnotesize
\label{attrtab:receiverhrtf}
Attributes of receiver element {\bf hrtf}, inheriting from \hyperref[attrtab:receiver]{{\bf receiver}}\nopagebreak

\begin{tabularx}{\textwidth}{lXl}
\hline
name & description (type, unit) & def.\\
\hline
\hline
\indattr{Q\_notch} & quality factor of the notch filter (float) & 2.3\\
\hline
\indattr{alphamin} & parameter which determines the depth of the high-shelf realizing the SHM (float) & 0.14\\
\hline
\indattr{alphamin\_front} & parameter which determines the depth of the second high-shelf (float) & 0.39\\
\hline
\indattr{alphamin\_up} & parameter which determines the depth of the second high-shelf (float) & 0.1\\
\hline
\indattr{angle} & Position of the ears on the sphere (float, deg) & 90\\
\hline
\indattr{c} & Speed of sound (float, m/s) & 340\\
\hline
\indattr{decorr} & Flag to use decorrelation of diffuse sounds (bool) & false\\
\hline
\indattr{decorr\_length} & Decorrelation length (float, s) & 0.05\\
\hline
\indattr{diffuse\_hrtf} & apply hrtf model also to diffuse rendering (bool) & false\\
\hline
\indattr{freq\_end} & notch center frequency at [0 0 1] (float, Hz) & 650\\
\hline
\indattr{freq\_start} & notch center frequency at \attr{startangle\_notch} (float, Hz) & 1300\\
\hline
\indattr{gaincorr} & channel-wise gain correction (float array, dB) & 0 0\\
\hline
\indattr{maxgain} & gain applied at [0 0 1] -- gain is 0 dB at \attr{startangle\_notch} and increases linearly (float, dB) & -5.4\\
\hline
\indattr{omega} & cut-off frequency of the high-self realizing the SHM (float, Hz) & 3100\\
\hline
\indattr{omega\_front} & cut-off frequency of the second high-self (float, Hz) & 11200\\
\hline
\indattr{omega\_up} & cut-off frequency of the second high-shelf in Hz (float, Hz) & 2125\\
\hline
\indattr{prewarpingmode} & Azimuth pre-warping mode, 0 = original, 1 = none, 2 = corrected (uint32) & 0\\
\hline
\indattr{radius} & Radius of sphere modeling the head (float, m) & 0.08\\
\hline
\indattr{sincorder} & Sinc interpolation order of ITD delay line (uint32) & 0\\
\hline
\indattr{sincsampling} & Sinc table sampling of ITD delay line, or 0 for no table. (uint32) & 64\\
\hline
\indattr{startangle\_front} & the second high-shelf, e.g. to model pinna shadow effect, is applied when the angle with respect to front direction [1 0 0] is larger than \attr{startangle\_front} (float, deg) & 0\\
\hline
\indattr{startangle\_notch} & notch filter to model concha notch is applied if angle with respect to up direction [0 0 1] is smaller than \attr{startangle\_notch} (float, deg) & 102\\
\hline
\indattr{startangle\_up} & the third high-shelf which models the shadow effect of the torso is applied when the angle with respect to up direction [0 0 1] is larger than \attr{startangle\_up} (float, deg) & 135\\
\hline
\indattr{thetamin} & angle with respect to the position of the ears at which the maximum depth of the high-shelf realizing the SHM is reached (float, deg) & 160\\
\hline
\end{tabularx}
}
\end{snugshade}


\begin{tscattributes}
\indattr{sincorder}         & Sinc interpolation order of ITD delay line (0)                                    \\
\indattr{c}                 & Speed of sound in m/s (340)                                                       \\
\indattr{decorr\_length}    & Decorrelation length in seconds (0.05)                                            \\
\indattr{decorr}            & Flag to use decorrelation of diffuse sounds (false)                               \\
\indattr{radius}            & Radius of sphere modeling the head (0.08)                                        \\
\indattr{angle}             & Position of the ears on the sphere in degrees (90)                                 \\
\indattr{thetamin}          & angle with respect to the position of the ears at which
the maximum depth of the high-shelf realizing the SHM is reached in degrees (160)                                \\
\indattr{alphamin}          & parameter which determines the depth of the high-shelf
realizing the SHM (0.14)                                                                                        \\
\indattr{omega}             & cut-off frequency of the high-self realizing the SHM in Hz (3100)                 \\
\indattr{startangle\_front} & the second high-shelf, e.g. to model pinna shadow effect,
is applied when the angle with respect to front direction [1 0 0] is larger than
\indattr{startangle\_front} in degrees (0)                                                                       \\
\indattr{alphamin\_front}   & parameter which determines the depth of the second high-shelf (0.39)               \\
\indattr{omega\_front}      & cut-off frequency of the second high-self in Hz (11200)                           \\
\indattr{startangle\_up}    & the third high-shelf which models the shadow effect of
the torso is applied when the angle with respect to up direction [0 0 1] is larger
than \indattr{startangle\_up} in degrees (135)                                                                   \\
\indattr{omega\_up}         & cut-off frequency of the second high-shelf in Hz (\indattr{c}/\indattr{radius}/2) \\
\indattr{alphamin\_up}      & parameter which determines the depth of the second high-shelf (0.1)                \\
\indattr{startangle\_notch} & notch filter to model concha notch is applied if angle with
respect to up direction [0 0 1] is smaller than \indattr{startangle\_notch} in degrees (102)                     \\
\indattr{freq\_start}       & notch center frequency at \indattr{startangle\_notch} in Hz (1300)                \\
\indattr{freq\_end}         & notch center frequency at [0 0 1] in Hz (650)                                     \\
\indattr{maxgain}           & gain applied at [0 0 1] in dB (-5.4) - gain is 0 dB at
\indattr{startangle\_notch} and increases linearly                                                              \\
\indattr{Q\_notch}          & quality factor of the notch filter (2.3)                                          \\
\end{tscattributes}
