\index{ORTF stereo microphone}\index{stereo}
 
This receiver implements a classic ORTF stereo microphone
technique. The cardioid microphone characteristic is frequency
dependent; the 6 dB cut-off frequency for 90 degrees is specified by
the attribute \indattr{f6db}. The attribute \indattr{fmin} defines the
cut-off frequency for sources from 180 degrees angle of incidence. To
disable the frequency dependence and use a broadband cardioid polar
pattern instead, use the attribute \attr{broadband="tue"}. The
attributes \indattr{distance} and \indattr{angle} control the
microphone geometry.

Typical values for small diaphragm microphones are \attr{f6db="3000"}
and \attr{fmin="800"} (these are the default values since version
0.172.2); for higher directivity use \attr{f6db="1000"}
and \attr{fmin="60"} (default values for earlier versions).

\definecolor{shadecolor}{RGB}{255,230,204}\begin{snugshade}
{\footnotesize
\label{attrtab:receiverortf}
Attributes of receiver element {\bf ortf}, inheriting from \hyperref[attrtab:receiver]{{\bf receiver}}\nopagebreak

\begin{tabularx}{\textwidth}{l>{\raggedright}XX}
\hline
name & description (type, unit) & def.\\
\hline
\hline
\indattr{angle} & Angular distance between microphone axes(double, deg) & 110\\
\hline
\indattr{attscale} & Scaling factor for cosine attenuation function(double) & 1\\
\hline
\indattr{broadband} & Use broadband cardioid characteristics(bool) & false\\
\hline
\indattr{c} & Speed of sound(double, m/s) & 340\\
\hline
\indattr{decorr} & Flag to use decorrelatin of diffuse sounds(bool) & false\\
\hline
\indattr{decorr\_length} & Decorrelation length(double, s) & 0.05\\
\hline
\indattr{distance} & Microphone distance(double, m) & 0.17\\
\hline
\indattr{f6db} & 6 dB cutoff frequency for 90 degrees(double, Hz) & 3000\\
\hline
\indattr{fmin} & Cutoff frequency for 180 degrees sounds(double, Hz) & 800\\
\hline
\indattr{sincorder} & Sinc interpolation order of ITD delay line(uint32) & 0\\
\hline
\indattr{sincsampling} & Sinc table sampling of ITD delay line, or 0 for no table.(uint32) & 64\\
\hline
\end{tabularx}
}
\end{snugshade}


\definecolor{shadecolor}{RGB}{236,236,255}\begin{snugshade}
{\footnotesize
\label{osctab:receivermodortf}
OSC variables:
\nopagebreak

\begin{tabularx}{\textwidth}{llllX}
\hline
path & fmt. & range & readable & description\\
\hline
\attr{/.../angle} & f &  & yes & Angular distance between microphone axes, in degree\\
\attr{/.../attscale} & f &  & yes & Scaling factor for cosine attenuation function\\
\attr{/.../decorr} & i & bool & yes & Flag to use decorrelatin of diffuse sounds\\
\attr{/.../distance} & f &  & yes & Microphone distance, in m\\
\hline
\end{tabularx}
}
\end{snugshade}
\definecolor{shadecolor}{RGB}{255,230,204}


Example:

\begin{lstlisting}[numbers=none]
<receiver type="ortf" f6db="3000" fmin="80" distance="0.17" angle="110"/>
\end{lstlisting}

