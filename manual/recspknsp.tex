Nearest speaker selection, i.e., always a single speaker is used to
render a virtual sound source. In case of moving sources or receivers,
the transition between two speakers will be linearly interpolated
within one audio block.

\begin{lstlisting}[numbers=none]
<receiver type="nsp"><speaker az="0"/>...</receiver>
\end{lstlisting}
%
This receiver also requires defining the position of the playback
channels and we can do it in the following way (see
\verb!example_nearest.tsc!):
\tscexample[linerange={4-7},firstnumber=4]{example_nearest}

\tscexampleext[linerange={2-7},firstnumber=2]{nsp.spk}

If we load a scene with such a receiver in \tascar{}, we will see all the
specified channels as an output of the rendering stage in the Jack
Audio. 
%
However, this time, for each source there is only one channel which is
active, i.e. the one for which there is the lowest angular distance
from the loudspeaker to the source.

The attribute \attr{useall} activates all speakers independent of the
source position.

\definecolor{shadecolor}{RGB}{255,230,204}\begin{snugshade}
{\footnotesize
\label{attrtab:receivernsp}
Attributes of receiver element {\bf nsp}, inheriting from \hyperref[attrtab:receiver]{{\bf receiver}} \hyperref[attrtab:speakerbased]{{\bf speakerbased}}\nopagebreak

\begin{tabularx}{\textwidth}{l>{\raggedright}XX}
\hline
name & description (type, unit) & def.\\
\hline
\hline
\indattr{useall} & activate all speakers independent of source position (bool) & false\\
\hline
\end{tabularx}
}
\end{snugshade}


\begin{snugshade}
{\footnotesize
\label{osctab:receivermodnsp}
OSC variables:
\nopagebreak

\begin{tabularx}{\textwidth}{llllX}
\hline
path & fmt. & range & r. & description\\
\hline
/scene/nsp/useall & i & bool & yes & \\
\hline
\end{tabularx}
}
\end{snugshade}

